\documentclass{article}

% if you need to pass options to natbib, use, e.g.:
%     \PassOptionsToPackage{numbers, compress}{natbib}
% before loading neurips_2020

% ready for submission
% \usepackage{neurips_2020}

% to compile a preprint version, e.g., for submission to arXiv, add add the
% [preprint] option:
    % \usepackage[preprint,nonatbib]{neurips_2020}

% to compile a camera-ready version, add the [final] option, e.g.:
    \usepackage[final]{neurips_2020}

% to avoid loading the natbib package, add option nonatbib:
    %  \usepackage[nonatbib]{neurips_2020}

\usepackage{graphics}
\usepackage{textcomp}
\usepackage{graphicx}
\usepackage{wrapfig}
\usepackage{amsfonts}
\usepackage{algorithm,algorithmicx}
\usepackage{algpseudocode}
\usepackage{microtype}      % microtypography
\usepackage{pdfpages}
\usepackage{amsmath}
\usepackage{amssymb}
\usepackage{float}
\usepackage{amsthm}
\usepackage{multicol}
\usepackage[toc,page]{appendix}
\usepackage{balance} % for balancing columns on the final page
\usepackage{caption} 
\usepackage{hyperref}
\hypersetup{
    colorlinks=true,
    linkcolor=blue,
    urlcolor=blue,
    citecolor=blue
}

\newtheorem{innercustomgeneric}{\customgenericname}
\providecommand{\customgenericname}{}
\newcommand{\newcustomtheorem}[2]{%
  \newenvironment{#1}[1]
  {%
   \renewcommand\customgenericname{#2}%
   \renewcommand\theinnercustomgeneric{##1}%
   \innercustomgeneric
  }
  {\endinnercustomgeneric}
}

\newcustomtheorem{customthm}{Theorem}
\newcustomtheorem{customlemma}{Lemma}

\title{Cooperation in Multi-Agent Learning}

% The \author macro works with any number of authors. There are two commands
% used to separate the names and addresses of multiple authors: \And and \AND.
%
% Using \And between authors leaves it to LaTeX to determine where to break the
% lines. Using \AND forces a line break at that point. So, if LaTeX puts 3 of 4
% authors names on the first line, and the last on the second line, try using
% \AND instead of \And before the third author name.

\author{
  Karush Suri, Lacra Pavel\\
   Department of Electrical \& Computer Engineering, University of Toronto, Canada.\\
  \texttt{karush.suri@mail.utoronto.ca}
}


\begin{document}

\maketitle

\begin{abstract}
Advancements in Multi-Agent Reinforcement Learning (MARL) are motivated by cooperation in agents arising from Game Theory (GT). Agents must collaborate in practical scenarios in order to achieve complex objectives and attain strategies which depict optimal behavior. The need for cooperation is further highlighted in the case of partially-observed settings wherein agents have restricted access to environment observations. We revisit cooperation in MARL from the viewpoint of GT and stochastic dynamics of environments. The contributions of our work are twofold. (1) We analyze and demonstrate the effectiveness of cooperative MARL in the case of complex and partially-observed tasks consisting of high-dimensional action spaces and stochastic dynamics. (2) We leverage the empirical demonstrations to construct a novel optimization objective which addresses the detrimental effects of spurious states across agents. Our large-scale experiments carried out on the StarCraft II benchmark depict the effectiveness of cooperative MARL and our novel objective for obtaining optimal strategies under stochastic dynamics. 

\end{abstract}

\begin{multicols}{2}

\section{Introduction}
jdfvje
% High level content of GT and MARL
% the problem and what you are doing to fix it

%%%%%%%%%%%%%%%%%%%%%%%%%%%%%%%%%%%%%%%%%%%%%%%%%%%%%%%%%%%%%%%%%%%%%%%%

\section{Related Work}

\subsection{Learning in Games}
% lit review of GT (must cover everything)

\subsection{Multi-Agent Learning}
% lit review of MARL (must cover everything)

%%%%%%%%%%%%%%%%%%%%%%%%%%%%%%%%%%%%%%%%%%%%%%%%%%%%%%%%%%%%%%%%%%%%%%%%

\section{Preliminaries}

\subsection{Stochastic Markov Games}
% explain markov games and their details

\subsection{Q-Learning}
% explain Q-learning in detail

\subsection{Multi-Agent Learning}
% explain MARL in detail

%%%%%%%%%%%%%%%%%%%%%%%%%%%%%%%%%%%%%%%%%%%%%%%%%%%%%%%%%%%%%%%%%%%%%%%%

\section{Cooperation in Multi-Agent Learning}
% talk about cooperation, implementation and analyze

\subsection{The Partial Observability Setting}
% explain the partial observability setting in detail

\subsection{Learning Model-Free Behaviors}
% expand out each algorithm with its details

%%%%%%%%%%%%%%%%%%%%%%%%%%%%%%%%%%%%%%%%%%%%%%%%%%%%%%%%%%%%%%%%%%%%%%%%
\section{Tackling Spurious Dynamics}
% novelty comes here, talk about surprise 
% talk about model-based and model-free methods
% highlght the math- ATLEAST 2 NOVEL CONTRIBUTIONS!!
% do no dive too much into RL, think from a GT perspective


%%%%%%%%%%%%%%%%%%%%%%%%%%%%%%%%%%%%%%%%%%%%%%%%%%%%%%%%%%%%%%%%%%%%%%%%


\section{Experiments}

\subsection{The StarCraft II Benchmark}
% explain out the SC2 benchmark, reasons and its highlights

\subsection{Performance}
% your RL results and details come here

\subsection{Spurious Dynamics}
% your novel results and explaination comes here

%%%%%%%%%%%%%%%%%%%%%%%%%%%%%%%%%%%%%%%%%%%%%%%%%%%%%%%%%%%%%%%%%%%%%%%%

\section{Conclusion}


\bibliographystyle{unsrt} 
\small{\bibliography{sample}}
\end{multicols}

% THINGS TO INCLUDE IN THE SUPPLEMENTARY-
% APPENDIX- derivations, additional results, implementation details
% VIDEOS- videos of all agents
% CODE- codebase for simulation

\end{document}
